\documentclass[11pt]{article}

%% PACKAGES
\usepackage{graphicx}
\usepackage{verbatim}
\usepackage{url}
\usepackage[printonlyused]{acronym}
\usepackage[ruled]{algorithm}
\usepackage{amsmath,amssymb,amsfonts,amsthm}
\usepackage{overpic}
\usepackage{calc}
\usepackage{color}
%\usepackage{times}
%\usepackage{ragged2e}
 \usepackage[margin=1.0in]{geometry}
\usepackage[colorlinks=false]{hyperref}
\usepackage{textcomp}
\usepackage{cite}
\usepackage{mdwlist}
\usepackage{subfiles}
\usepackage{enumitem}
\usepackage{calc}
\usepackage{array}
\usepackage{units}
\usepackage{arydshln,leftidx,mathtools}
\usepackage[caption=false,font=footnotesize]{subfig}
\usepackage{relsize}
\usepackage{float}
\usepackage{makecell}

\usepackage{algorithm}
\usepackage[noend]{algpseudocode}

\makeatletter
\let\@tmp\@xfloat     
\usepackage{fixltx2e}
\let\@xfloat\@tmp                    
\makeatother

\usepackage[subfigure]{tocloft}
\usepackage[singlespacing]{setspace}
%\usepackage[nodisplayskipstretch]{setspace}
%\setstretch{1.0}

%\renewcommand\cftsecafterpnum{\vskip\baselineskip}
%\renewcommand\cftsubsecafterpnum{\vskip\baselineskip}
%\renewcommand\cftsubsubsecafterpnum{\vskip\baselineskip}

%\usepackage{mathtools}
%\usepackage[framed]{mcode}

\usepackage{pgfplots}

\usepackage{cancel}

\usepackage{tikz}
\usetikzlibrary{calc,patterns,decorations.pathmorphing,decorations.markings,fit,backgrounds}

\usepackage[strict]{changepage} %use to manually place figs/tables to get them within the margins

\makeatletter
\g@addto@macro\normalsize{%
  \setlength\abovedisplayskip{0.25pt}
  \setlength\belowdisplayskip{0.25pt}
  \setlength\abovedisplayshortskip{0.25pt}
  \setlength\belowdisplayshortskip{0.25pt}
}
\makeatother



\setlength{\parskip}{\baselineskip}

%% GRAPHICS PATH
\graphicspath{{../../../shared_latex_inputs/images}}

%% TODO
\newcommand{\todo}[1]{\vspace{5 mm}\par \noindent \framebox{\begin{minipage}[c]{0.98 \columnwidth} \ttfamily\flushleft \textcolor{red}{#1}\end{minipage}}\vspace{5 mm}\par}

%% MACROS
\providecommand{\abs}[1]{\lvert#1\rvert}
\providecommand{\norm}[1]{\lVert#1\rVert}
\providecommand{\dualnorm}[1]{\norm{#1}_\ast}
\providecommand{\set}[1]{\lbrace\,#1\,\rbrace}
\providecommand{\cset}[2]{\lbrace\,{#1}\nobreak\mid\nobreak{#2}\,\rbrace}
\providecommand{\onevect}{\mathbf{1}}
\providecommand{\zerovect}{\mathbf{0}}
\providecommand{\field}[1]{\mathbb{#1}}
\providecommand{\C}{\field{C}}
\providecommand{\R}{\field{R}}
\providecommand{\polar}{\triangle}
\providecommand{\Cspace}{\mathcal{Q}}
\providecommand{\Fspace}{\mathcal{F}}
\providecommand{\free}{\text{\{}\mathsf{free}\text{\}}}
\providecommand{\iff}{\Leftrightarrow}
\providecommand{\qstart}{q_\text{initial}}
\providecommand{\qgoal}{q_\text{final}}
\providecommand{\contact}[1]{\Cspace_{#1}}
\providecommand{\feasible}[1]{\Fspace_{#1}}
\providecommand{\prob}[2]{p(#1|#2)}
\providecommand{\prior}[1]{p(#1)}
\providecommand{\Prob}[2]{P(#1|#2)}
\providecommand{\Prior}[1]{P(#1)}
\providecommand{\parenth}[1] {\left(#1\right)}
\providecommand{\braces}[1] {\left\{#1\right\}}
\providecommand{\micron}{\hbox{\textmu m}}

%% MATH FUNCTION NAMES
\DeclareMathOperator{\conv}{conv}
\DeclareMathOperator{\cone}{cone}
\DeclareMathOperator{\homog}{homog}
\DeclareMathOperator{\domain}{dom}
\DeclareMathOperator{\range}{range}
\DeclareMathOperator{\argmax}{arg\,max}
\DeclareMathOperator{\argmin}{arg\,min}
\DeclareMathOperator{\area}{area}
\DeclareMathOperator{\sign}{sign}
\DeclareMathOperator{\mathspan}{span}
\DeclareMathOperator{\sn}{sn}
\DeclareMathOperator{\cn}{cn}
\DeclareMathOperator{\dn}{dn}
\DeclareMathOperator*{\minimize}{minimize}

\DeclareMathOperator{\atan2}{atan2}

\newtheorem{theorem}{Theorem}
\newtheorem{lemma}[theorem]{Lemma}

%\setlength{\RaggedRightParindent}{2em}
%\setlength{\RaggedRightRightskip}{0pt plus 3em}
%\pagestyle{empty}


\title{{\Huge MGAnDSMs Maneuver Monte Carlo Utility}}
\vspace{0.5cm}
\author{Donald H. Ellison \thanks{Aerospace Engineer, NASA Goddard Space Flight Center, Flight Dynamics and Mission Design Branch Code 595}}
\vspace{0.5cm}

\date{}

\begin{document}

\begin{titlepage}
\maketitle
%\thispagestyle{empty}
\begin{table}[H]
	\centering
	\begin{tabular}{|l|l|l|}
		\hline
		\textbf{Revision Date} & \textbf{Author} & \textbf{Description of Change} \\ \hline
		\date{January 18, 2019} & Donald Ellison & Created document, initial description of software utility \\
		\hline
	\end{tabular}
\end{table}
\end{titlepage}



\newpage
\tableofcontents
\thispagestyle{empty}
\newpage

\clearpage
\setcounter{page}{1}




\section*{List of Acronyms}
\begin{acronym}
%To define the acronym and include it in the list of acronyms: \acro{acronym}{definition}
%To define the acronym and exclude it from the list of acronyms:  \acro{acronym}{definition}
%
%\ac{acronym} Expand and identify the acronym the first time; use only the acronym thereafter
%\acf{acronym} Use the full name of the acronym.
%\{acronym} Use the acronym, even before the first corresponding \ac command
%\acl{acronym}  Expand the acronym without using the acronym itself.
%
%

\acro{ACS}{attitude control system}
\acro{ACO}{Ant Colony Optimization}
\acro{AD}{Automatic Differentiation}
\acro{ADL}{Architecture Design Laboratory}
\acro{AES}{Advanced Exploration Systems}
\acro{AGA}{aerogravity assist}
\acro{ALARA}{As Low As Reasonably Achievable}
\acro{API}{application programming interface}
\acro{BB}{branch and bound}
\acro{BVP}{Boundary Value Problem}
\acro{CATO}{Computer Algorithm for Trajectory Optimization}
\acro{CL}{confidence level}
\acro{CONOPS}{concept of operations}
\acro{COV}{Calculus of Variations}
\acro{D/AV}{Descent/Ascent Vehicle}
\acro{DE}{Differential Evolution}
\acro{DLA}{Declination of Launch Asymptote}
\acro{RLA}{Right Ascension of Launch Asymptote}
\acro{RA}{right ascension}
\acro{DEC}{declination}
\acro{DPTRAJ/ODP}{Double Precision Trajectory and Orbit Determination Program}
\acro{DSH}{Deep Space Habitat}
\acro{DSN}{Deep Space Network}
\acro{DSMPGA}{Dynamic-Size Multiple Population Genetic Algorithm}
\acro{EB}{Evolutionary Branching}
\acro{ECLSS}{environmental control and life support system}
\acro{ELV}{expendable launch vehicle}
\acro{EMME}{Earth to Mars, Mars to Earth}
\acro{EMMVE}{Earth to Mars, Mars to Venus to Earth}
\acro{EMTG}{Evolutionary Mission Trajectory Generator}
\acro{EVMME}{Earth to Venus to Mars, Mars to Earth}
\acro{EVMMVE}{Earth to Venus to Mars, Mars to Venus to Earth}
\acro{ERRV}{Earth Return Re-entry Vehicle}
\acro{FISO}{Future In-Space Operations}
\acro{FMT}{Fast Mars Transfer}
\acro{GASP}{Gravity Assist Space Pruning}
\acro{GCR}{galactic cosmic radiation}
\acro{GRASP}{Greedy Randomized Adaptive Search Procedure}
\acro{GSFC}{Goddard Space Flight Center}
\acro{GTOC}{Global Trajectory Optimization Competition}
\acro{GTOP}{Global Trajectory Optimization Problem}
\acro{HAT}{Human Architecture Team}
\acro{HGGA}{Hidden Genes Genetic Algorithm}
\acro{IMLEO}{Initial Mass in \acl{LEO}}
\acro{IPOPT}{Interior Point OPTimizer}
\acro{ISS}{International Space Station}
\acro{JHUAPL}{Johns Hopkins University Applied Physics Laboratory}
\acro{JSC}{Johnson Space Center}
\acro{KKT}{Karush-Kuhn-Tucker}
\acro{LEO}{Low Earth Orbit}
\acro{LRTS}{lazy race tree search}
\acro{MONTE}{Mission analysis, Operations, and Navigation Toolkit Environment}
\acro{MCTS}{Monte Carlo tree search}
\acro{MGA}{Multiple Gravity Assist}
\acro{MIRAGE}{Multiple Interferometric Ranging Analysis using GPS Ensemble}
\acro{MOGA}{Multi-Objective Genetic Algorithm}
\acro{MOSES}{Multiple Orbit Satellite Encounter Software}
\acro{MPI}{message passing interface}
\acro{MPLM}{Multi-Purpose Logistics Module}
\acro{MSFC}{Marshall Space Flight Center}
\acro{NELLS}{NASA Exhaustive Lambert Lattice Search}
\acro{NSGA}{Non-Dominated Sorting Genetic Algorithm}
\acro{NSGA-II}{Non-Dominated Sorting Genetic Algorithm II}
\acro{NHATS}{Near-Earth Object Human Space Flight Accessible Targets Study}
\acro{NTP}{Nuclear Thermal Propulsion}
\acro{OD}{orbit determination}
\acro{OOS}{On-Orbit Staging}
\acro{PCC}{Pork Chop Contour}
\acro{PEL}{permissible exposure limits}
\acro{PLATO}{PLAnetary Trajectory Optimization}
\acro{REID}{risk of exposure-induced death}
\acro{RTBP}{Restricted Three Body Problem}
\acro{SA}{Simulated Annealing}
\acro{SLS}{Space Launch System}
\acro{SNOPT}{Sparse Nonlinear OPTimizer}
\acro{SOI}{sphere of influence}
\acro{SPE}{solar particle events}
\acro{SQP}{sequential quadratic programming}
\acro{SRAG}{Space Radiation Analysis Group}
\acro{TEI}{Trans-Earth Injection}
\acro{TOF}{time of flight}
\acro{TPBVP}{Two Point Boundary Value Problem}
\acro{TMI}{Trans-Mars Injection}
\acro{VARITOP}{Variational calculus Trajectory Optimization Program}
\acro{VILM}{v-infinity leveraging maneuver}
\acro{MOI}{Mar Orbit Injection}
\acro{PCM}{Pressurized Cargo Module}
\acro{STS}{Space Transportation System}
\acro{EDS}{Earth Departure Stage}
\acro{NEO}{near-Earth asteroid}
\acro{IDC}{Integrated Design Center}
\acro{SEP}{solar-electric propulsion}
\acro{SRP}{solar radiation pressure}
\acro{NEP}{nuclear-electric propulsion}
\acro{REP}{radioisotope-electric propulsion}
\acro{DRM}{Design Reference Missions}

\acro{EDL}{entry, descent, and landing}
\acro{ASCII}{American Standard Code for Information Interchange}
\acro{AU}{Astronomical Unit}
\acro{BWG}{Beam Waveguides}
\acro{CCB}{Configuration Control Board}
\acro{CMO}{Configuration Management Office}
\acro{CODATA}{Committee on Data for Science and Technology}
\acro{DEEVE}{Dynamically Equivalent Equal Volume Ellipsoid}
\acro{DRA}{Design Reference Asteroid}
\acro{EME2000}{Earth Centered, Earth Mean Equator and Equinox of J2000 (Coordinate Frame)}
\acro{EOP}{Earth Orientation Parameters}
\acro{ET}{Ephemeris Time}
\acro{FDS}{Flight Dynamics System}
\acro{FTP}{File Transfer Protocol}
\acro{GSFC}{Goddard Space Flight Center}
\acro{PI}{Principal Investigator}
\acro{HEF}{High Efficiency}
\acro{IAG}{International Association of Geodesy}
\acro{IAU}{International Astronomical Union}
\acro{IERS}{International Earth Rotation and Reference Systems Service}
\acro{ICRF}{International Celestial Reference Frame}
\acro{ITRF}{International Terrestrial Reference System}
\acro{IOM}{Interoffice Memorandum}
\acro{JD}{Julian Date}
\acro{JPL}{Jet Propulsion Laboratory}
\acro{LM}{Lockheed Martin}
%\acro{LP150Q}{}
%\acros{LP100K}{}
\acro{MAVEN}{Mars Atmosphere and Volatile EvolutioN}
\acro{MJD}{Modified Julian Date}
\acro{MOID}{Minimum Orbit Intersection Distance}
\acro{MPC}{Minor Planet Center}
\acro{NASA}{National Aeronautics and Space Administration}
\acro{NDOSL}{\ac{NASA} Directory of Station Locations}
\acro{NEA}{near-Earth asteroid}
\acro{NEO}{near-Earth object}
\acro{NIO}{Nav IO}
\acro{OSIRIS-REx}{Origins Spectral Interpretation Resource Identification Security-Regolith Explorer}
\acro{PHA}{Potentially Hazardous Asteroid}
\acro{PHO}{Potentially Hazardous Object}
\acro{SBDB}{Small-Body Database}
\acro{SI}{International System of Units}
\acro{SPICE}{Spacecraft Planet Instrument Camera-matrix Events}
\acro{SPK}{SPICE Kernel}
\acro{SRC}{Sample Return Capsule}
\acro{SSD}{Solar System Dynamics}
\acro{STK}{Systems Tool Kit}
\acro{TAI}{International Atomic Time}
\acro{TBD}{To Be Determined}
\acro{TBR}{To Be Reviewed}
\acro{TCB}{Barycentric Coordinate Time}
\acro{TDB}{Temps Dynamiques Barycentrique, Barycentric Dynamical Time}
\acro{TDT}{Terrestrial Dynamical Time}
\acro{TT}{Terrestrial Time}
\acro{URL}{Uniform Resource Locator}
\acro{UT}{Universal Time}
\acro{UT1}{Universal Time Corrected for Polar Motion}
\acro{UTC}{Coordinated Universal Time}
\acro{USNO}{U. S. Naval Observatory}
\acro{YORP}{Yarkovsky-O'Keefe-Radzievskii-Paddack}

\acro{NLP}{nonlinear program}
\acro{MBH}{monotonic basin hopping}
\acro{MBH-C}{monotonic basin hopping with Cauchy hops}
\acro{FBS}{forward-backward shooting}
\acro{MGALT}{Multiple Gravity Assist with Low-Thrust}
\acro{MGALTS}{Multiple Gravity Assist with Low-Thrust using the Sundman transformation}
\acro{MGA-1DSM}{Multiple Gravity Assist with One Deep Space Maneuver}
\acro{MGAnDSMs}{Multiple Gravity Assist with \textit{n} Deep-Space Maneuvers using Shooting}
\acro{PSFB}{Parallel Shooting with Finite-Burn}
\acro{PSBI}{Parallel Shooting with Bounded Impulses}
\acro{FBLT}{Finite-Burn Low-Thrust}
\acro{FBLTS}{Finite-Burn Low-Thrust using the Sundman transformation}
\acro{ESA}{European Space Agency}
\acro{ACT}{Advanced Concepts Team}
\acro{IRAD}{independent research and development}
\acro{Isp}[$\text{I}_{sp}$]{specific impulse}
\acro{C3}[$C_3$]{hyperbolic excess energy}
\acro{GA}{genetic algorithm}
\acro{GALLOP}{ Gravity Assisted Low-thrust Local Optimization Program}
\acro{MALTO}{Mission Analysis Low-Thrust Optimization}
\acro{PaGMO}{Parallel Global Multiobjective Optimizer}
\acro{FRA}{feasible region analysis}
\acro{CP}{conditional penalty}
\acro{HOC}{hybrid optimal control}
\acro{HOCP}{hybrid optimal control problem}
\acro{PSO}{particle swarm optimization}
\acro{SEPTOP}{Solar Electric Propulsion Trajectory Optimization Program}
\acro{STOUR}{Satellite Tour Design Program}
\acro{STOUR-LTGA}{Satellite Tour Design Program - Low Thrust, Gravity Assist}
\acro{PaGMO}{Parallel Global Multiobjective Optimizer}
\acro{SDC}{static/dynamic control}
\acro{DDP}{Differential Dynamic Programming}
\acro{HDDP}{Hybrid Differential Dynamic Programming}
\acro{ACT}{Advanced Concepts Team}
\acro{GMAT}{General Mission Analysis Toolkit}
\acro{BOL}{beginning of life}
\acro{EOL}{end of life}
\acro{KSC}{Kennedy Space Center}
\acro{VSI}{variable \ac{Isp}}
\acro{RTG}{radioisotope thermal generator}
\acro{ASRG}{advanced Stirling radiosotope generator}
\acro{ARRM}{Asteroid Robotic Redirect Mission}
\acro{AATS}{Alternative Architecture Trade Study}
\acro{PPU}{power processing unit}
\acro{STM}{state transition matrix}
\acro{MTM}{maneuver transition matrix}
\acro{HPTM}{half-phase transition matrix}
\acro{BCI}{body-centered inertial}
\acro{BCF}{body-centered fixed}
\acro{UTTR}{Utah Test and Training Range}
\acro{EPV}{equatorial projection of $\mathbf{v}_\infty$}
\acro{KBO}{Kuiper belt object}
\acro{DSM}{deep-space maneuver}
\acro{BPT}{body-probe-thrust}
\acro{4PL}{four parameter logistic}
\acro{BCF}{body-centered fixed}
\acro{COE}{classical orbit elements}
\acro{GSL}{Gnu Scientific Library}
\acro{NEXT}{NASA's Evolutionary Xenon Thruster}

\acro{SMA}{semi-major axis}
\acro{ECC}{eccentricity}

\acro{GSAD}{Ghosh Sparse Algorithmic Differentiation}

\end{acronym}

% --------------------------------------------------------------------------------------------------------------------------
% --------------------------------------------------------------------------------------------------------------------------


\section{MGAnDSMs Missed Maneuver Analysis Strategy}
\label{sec:missedManeuverAnalysis}

At a high level, missed maneuver analysis for EMTG's chemical transcription (\ac{MGAnDSMs}) involves re-optimization of a mission-to-go starting from an off-nominal state at a critical \ac{DSM}. The initial condition can take the form of a full off-nominal 7-state generated from a navigation Monte Carlo analysis performed on the mission up to the point of the critical \ac{DSM}], or just the epoch of the delayed \ac{DSM}, where the initial state is obtained by propagating the pre-burn state of the spacecraft along the optimal trajectory for the length of the maneuver delay.

\section{MGAnDSMs Missed Maneuver Software Utility}

The Python utility $\mathtt{PyEMTG/SimpleMonteCarlo/MGAnDSMs\_MissedManeuver.py}$ can be used to perform mission-to-go re-optimization for an existing \ac{MGAnDSMs} mission that has been converged in EMTG. The program generates an EMTG options file based on an existing EMTG MissionOptions and Mission object pair. Besides those two inputs, the user must provide an identification string for the burn that they wish to perform the analysis on (e.g. 'j3p0b1'), an initial state to start the optimization from (including the state representation and frame), a burn delay time if the analysis is for a pure delay along the nominal trajectory, initial wait time bounds and a flag indicating whether or not a monoprop TCM is desired following the missed maneuver event.

\begin{itemize}
	\item $\mathtt{\mathbf{original\_mission}}$: EMTG mission object corresponding to the nominal trajectory
	\item $\mathtt{\mathbf{original\_options}}$: EMTG mission options object corresponding to the nominal trajectory
	\item $\mathtt{\mathbf{burn\_id\_string}}$: String identifying the maneuver of interest. Must be of the form $\mathtt{`jXpYbZ'}$
	\item $\mathtt{\mathbf{initial\_state}}$: Seven entry list containing initial spaceraft state $\left[x, y, z, vx, vy, vz, m, \text{epoch}\right]$
	\item $\mathtt{\mathbf{initial\_state\_representation}}$: Integer corresponding to EMTG's state representation enums. One of $\{0: \text{Cartesian}, 1: \text{SphericalRADEC}, 2: \text{SphericalAZFPA}\}$
	\item $\mathtt{\mathbf{initial\_state\_frame}}$: Integer corresponding to EMTG's frame enums. One of $\{0: \text{ICRF}, 1: \text{J2000\_BCI}, 2: \text{J2000\_BCF}, 3: \text{TrueOfDate\_BCI}, 4: \text{TrueOfDate\_BCF}, 5: \text{PrincipleAxes}, 6: \text{Topocentric}, 7: \text{Polar}\}$.
	\item $\mathtt{\mathbf{burn\_delay}}$: Float indicating amount of time to delay maneuver. If > 0.0, maneuver is delayed and the initial state is determined by propagating the nominal pre-\ac{DSM} position/velocity. If < 0.0, maneuver is early and the initial state is determined by propagating backwards from the pre-\ac{DSM} position/velocity. If 0.0, then the initial state and burn epoch is fully determined by $\mathtt{\mathbf{initial\_state}}$.
	\item $\mathtt{\mathbf{initial\_maneuver\_wait\_time\_bounds}}$: Two entry list containing the lower and upper bounds on the wait time allowed before performing the missed maneuver
	\item $\mathtt{\mathbf{include\_TCM}}$: Boolean, if true, a monoprop TCM opportunity is inserted after the missed maneuver, before any subsequent \ac{DSM}s and before the journey right-hand boundary.
\end{itemize}

\section{Constructing an EMTG Options Object for a Missed Maneuver Scenario}

The missed maneuver utility performs the following actions:

\begin{enumerate}
	\item Removes all JourneyOptions objects that occur prior to the missed maneuver event
	\item Sets the first journey's departure class to a FreePoint and its type to a launch/direct insertion
	\item Fixes the departure mass
	\item Includes the departure maneuver in the optimization cost function
	\item Sets the pre-departure maneuver state and frame
	\item Sets the departure launch window opening and first journey wait times
		\begin{enumerate}
			\item If a pure delay is applied, the wait time lower bound is set to the delay duration, and the upper is set to the second entry in $\mathtt{\mathbf{initial\_maneuver\_wait\_time\_bounds}}$
			\item If the delay is negative (i.e. the maneuver is happening early), then the upper wait time bound is set to the delay duration and the lower is set to the first entry in $\mathtt{\mathbf{initial\_maneuver\_wait\_time\_bounds}}$
			\item If a full 7-state is provided, then both wait time bounds are taken from $\mathtt{\mathbf{initial\_maneuver\_wait\_time\_bounds}}$
		\end{enumerate}
	\item Computes a new trialX initial guess based on the missed maneuver scenario, accounting for the truncated first journey
\end{enumerate}

Figure \ref{fig:MGAnDSMs_missed_maneuver} depicts a four-\ac{DSM} MGAnDSMs phase, where labels marked with a superscript asterisk refer to objects and quantities associated with the nominal trajectory. The lower image depicts a missed maneuver event for $\text{DSM}^F_1$. The dashed line could represent a delay along the nominal trajectory or a more general off-nominal 7-state. For any missed maneuver analysis, the new EMTG mission will begin with a launch/direct insertion from a FreePoint, with the missed maneuver now handled as the phase departure maneuver. The $\mathtt{MGAnDSMs\_MissedManeuver.py}$ utility offers the ability to insert a monopropellant \ac{TCM} (maneuver $DSM_{F_0}$ in Fig. \ref{fig:MGAnDSMs_missed_maneuver}) before the first nominal main \ac{DSM} by setting $\mathtt{\mathbf{include\_TCM}}$ to true. The new initial guess will place the \ac{TCM} at the half-way point between the departure maneuver and the first remaining nominal \ac{DSM}.


\begin{figure}[h!]
	\centering
	\includegraphics*[clip, trim=0.0cm 0.0cm 0.0cm 0.0cm, width=1.0\linewidth]{MGAnDSMsphase_Missed_Maneuver.png}
	\caption{\label{fig:MGAnDSMs_missed_maneuver} MGAnDSMs missed maneuver.}
\end{figure}

It is clear from Fig. \ref{fig:MGAnDSMs_missed_maneuver}, that the structure of the journey in which the missed maneuver occurs can change quite a bit. As a result, $\mathtt{MGAnDSMs\_MissedManeuver.py}$ automatically recomputes an initial guess based on the nominal trajectory solution contained in the original MissionOptions.trialX container. Specifically, $\mathtt{MGAnDSMs\_MissedManeuver.py}$ will transfer the original initial guesses for the remaining DSM components to the new guess (Fig. \ref{fig:ManeuverList}), recalculate all of the burn indices in the journey-to-go and adjust the initial guesses for the phase propellant usage variables.

\begin{figure}[H]
	\centering
	\includegraphics*[clip, trim=0.0cm 0.0cm 0.0cm 0.0cm, width=0.8\linewidth]{ManeuverList.png}
	\caption{\label{fig:ManeuverList} Transfer of maneuver initial guess information for the post-missed maneuver journey-to-go.}
\end{figure}

Computation of the new burn indices relies on two subroutines: $\mathtt{generateBurnIndexList}$ and $\mathtt{computeNewBurnIndices}$. The former takes the integer number of \ac{DSM}s allowed in an MGAnDSMs phase as an input and returns a list of burn index names for that phase in order starting at the left boundary and proceeding to the right boundary. The second subroutine returns a list of two-length lists where each two-length list contains the name of a journey-to-go burn index as the first entry and its floating point value as the second entry. The burn indices for the truncated first journey are re-computed using Eq. (\ref{eq:general_burn_index}), where $\Delta t_{\text{p}}$ is the flight time of the truncated journey-to-go and $\Delta t_{\text{p}}^*$ is the flight time of the entire nominal journey. Should the departure maneuver have some delay applied to it, then the first burn index $F_0$ is computed using Eq. (\ref{eq:departure_burn_index}).

\begin{equation}
\label{eq:general_burn_index}
\alpha = \alpha^* \frac{\Delta t_{\text{p}}^*}{\Delta t_{\text{p}}}
\end{equation}

\begin{equation}
\label{eq:departure_burn_index}
\alpha_{F_0} = \frac{\alpha^* \Delta t_{\text{p}}^* + \Delta t_{\text{depart}}}{\Delta t_{\text{p}}}
\end{equation}

\noindent Initial guesses for the propellant tank (propellant consumption) variables are computed using Eq. (\ref{eq:oxidizer_mass_drop}) and (\ref{eq:fuel_mass_drop}):

\begin{equation}
\label{eq:oxidizer_mass_drop}
 \Delta m_{o} = \sum_{k = 0}^{n - 1} m^{-}_k\left[1 - e^{\left(\frac{\Delta v_k}{g~I_{\text{sp}}}\right)}\right]\left(\frac{1}{1 + \frac{1}{r_{\text{m}}}}\right)
\end{equation}

\begin{equation}
\label{eq:fuel_mass_drop}
\Delta m_{f} = \sum_{k = 0}^{n - 1} m^{-}_k\left[1 - e^{\left(\frac{\Delta v_k}{g~I_{\text{sp}}}\right)}\right]\left(\frac{1}{1 + r_{\text{m}}}\right)
\end{equation},

\noindent where $m^-$ is the mass of the spacecraft immediately prior to the applied maneuver, $g$ is the standard acceleration due to gravity, $I_{\text{sp}}$ is the specific impulse of the chemical engine and $r_m$ is the mixture ratio.

%\bibliographystyle{AAS_publication}
%\bibliography{EMTGbib_Jacob_June_2014}



\end{document}





















