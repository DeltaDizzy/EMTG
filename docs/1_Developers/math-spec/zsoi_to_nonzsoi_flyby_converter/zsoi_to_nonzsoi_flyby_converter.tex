\documentclass[]{article}
\usepackage{nomencl}
\usepackage{hyperref}
\hypersetup{pdffitwindow=true,
	pdfpagemode=UseThumbs,
	breaklinks=true,
	colorlinks=true,
	linkcolor=black,
	citecolor=black,
	filecolor=black,
	urlcolor=black}

\usepackage{verbatim}
\usepackage[T1]{fontenc}
\usepackage{graphicx}%
\usepackage{amsmath}
\usepackage{amssymb}
\usepackage{amsthm}
\usepackage{subfigure}
\usepackage[makeroom]{cancel}
\usepackage{indentfirst}
\usepackage{color}
\usepackage{bm}
\usepackage{mathtools}

\usepackage{rotating}
\usepackage{comment}
\usepackage{here}
\usepackage{tabularx}
\usepackage{multirow}
\usepackage{setspace}
\usepackage{pdfpages}
\usepackage{float}
\usepackage[section]{placeins}
\usepackage[nolist,nohyperlinks]{acronym}
\usepackage{array}

% quotes
\newcommand{\quotes}[1]{``#1''}

% vectors and such
\newcommand{\vb}[1]{\bm{#1}} % bold
\newcommand{\vbd}[1]{\dot{\bm{#1}}} % dot
\newcommand{\vbdd}[1]{\ddot{\bm{#1}}} % double dot
\newcommand{\vbh}[1]{\hat{\bm{#1}}} % hat
\newcommand{\vbt}[1]{\tilde{\bm{#1}}} % tilde
\newcommand{\vbth}[1]{\hat{\tilde{\bm{#1}}}} % tilde hat
\newcommand{\ddt}[1]{\frac{\mathrm{d} #1}{\mathrm{d} t}} % time derivative
\newcommand{\pd}[2]{\frac{\partial #1}{\partial #2}} % partial derivative
\newcommand{\crossmat}[1]{\left\{ {#1} \right\}^{\times}} % crossmat
\newcommand{\xb}[0]{\vb{x}_b}
\newcommand{\xc}[0]{\vb{x}_c}
\newcommand{\diff}{\mathrm{d}}

% handle possessive acronyms
\makeatletter
\newcommand{\acposs}[1]{%
	\expandafter\ifx\csname AC@#1\endcsname\AC@used
	\acs{#1}'s%
	\else
	\aclu{#1}'s (\acs{#1})%
	\fi
}
\makeatother

% handle long acronyms that need to break across lines
\makeatletter
\renewcommand*\AC@acs[1]{%
	\expandafter\AC@get\csname fn@#1\endcsname\@firstoftwo{#1}}
\makeatother

\makenomenclature
\makeindex

%opening
\title{Converting a Zero-Sphere-of-Influence Flyby to a Non-Zero-Sphere-of-Influence Flyby in EMTG}
\author{Noble Hatten}

\begin{document}

\maketitle



\section*{List of Acronyms}
\begin{acronym}
%To define the acronym and include it in the list of acronyms: \acro{acronym}{definition}
%To define the acronym and exclude it from the list of acronyms:  \acro{acronym}{definition}
%
%\ac{acronym} Expand and identify the acronym the first time; use only the acronym thereafter
%\acf{acronym} Use the full name of the acronym.
%\{acronym} Use the acronym, even before the first corresponding \ac command
%\acl{acronym}  Expand the acronym without using the acronym itself.
%
%

\acro{ACS}{attitude control system}
\acro{ACO}{Ant Colony Optimization}
\acro{AD}{Automatic Differentiation}
\acro{ADL}{Architecture Design Laboratory}
\acro{AES}{Advanced Exploration Systems}
\acro{AGA}{aerogravity assist}
\acro{ALARA}{As Low As Reasonably Achievable}
\acro{API}{application programming interface}
\acro{BB}{branch and bound}
\acro{BVP}{Boundary Value Problem}
\acro{CATO}{Computer Algorithm for Trajectory Optimization}
\acro{CL}{confidence level}
\acro{CONOPS}{concept of operations}
\acro{COV}{Calculus of Variations}
\acro{D/AV}{Descent/Ascent Vehicle}
\acro{DE}{Differential Evolution}
\acro{DLA}{Declination of Launch Asymptote}
\acro{RLA}{Right Ascension of Launch Asymptote}
\acro{RA}{right ascension}
\acro{DEC}{declination}
\acro{DPTRAJ/ODP}{Double Precision Trajectory and Orbit Determination Program}
\acro{DSH}{Deep Space Habitat}
\acro{DSN}{Deep Space Network}
\acro{DSMPGA}{Dynamic-Size Multiple Population Genetic Algorithm}
\acro{EB}{Evolutionary Branching}
\acro{ECLSS}{environmental control and life support system}
\acro{ELV}{expendable launch vehicle}
\acro{EMME}{Earth to Mars, Mars to Earth}
\acro{EMMVE}{Earth to Mars, Mars to Venus to Earth}
\acro{EMTG}{Evolutionary Mission Trajectory Generator}
\acro{EVMME}{Earth to Venus to Mars, Mars to Earth}
\acro{EVMMVE}{Earth to Venus to Mars, Mars to Venus to Earth}
\acro{ERRV}{Earth Return Re-entry Vehicle}
\acro{FISO}{Future In-Space Operations}
\acro{FMT}{Fast Mars Transfer}
\acro{GASP}{Gravity Assist Space Pruning}
\acro{GCR}{galactic cosmic radiation}
\acro{GRASP}{Greedy Randomized Adaptive Search Procedure}
\acro{GSFC}{Goddard Space Flight Center}
\acro{GTOC}{Global Trajectory Optimization Competition}
\acro{GTOP}{Global Trajectory Optimization Problem}
\acro{HAT}{Human Architecture Team}
\acro{HGGA}{Hidden Genes Genetic Algorithm}
\acro{IMLEO}{Initial Mass in \acl{LEO}}
\acro{IPOPT}{Interior Point OPTimizer}
\acro{ISS}{International Space Station}
\acro{JHUAPL}{Johns Hopkins University Applied Physics Laboratory}
\acro{JSC}{Johnson Space Center}
\acro{KKT}{Karush-Kuhn-Tucker}
\acro{LEO}{Low Earth Orbit}
\acro{LRTS}{lazy race tree search}
\acro{MONTE}{Mission analysis, Operations, and Navigation Toolkit Environment}
\acro{MCTS}{Monte Carlo tree search}
\acro{MGA}{Multiple Gravity Assist}
\acro{MIRAGE}{Multiple Interferometric Ranging Analysis using GPS Ensemble}
\acro{MOGA}{Multi-Objective Genetic Algorithm}
\acro{MOSES}{Multiple Orbit Satellite Encounter Software}
\acro{MPI}{message passing interface}
\acro{MPLM}{Multi-Purpose Logistics Module}
\acro{MSFC}{Marshall Space Flight Center}
\acro{NELLS}{NASA Exhaustive Lambert Lattice Search}
\acro{NSGA}{Non-Dominated Sorting Genetic Algorithm}
\acro{NSGA-II}{Non-Dominated Sorting Genetic Algorithm II}
\acro{NHATS}{Near-Earth Object Human Space Flight Accessible Targets Study}
\acro{NTP}{Nuclear Thermal Propulsion}
\acro{OD}{orbit determination}
\acro{OOS}{On-Orbit Staging}
\acro{PCC}{Pork Chop Contour}
\acro{PEL}{permissible exposure limits}
\acro{PLATO}{PLAnetary Trajectory Optimization}
\acro{REID}{risk of exposure-induced death}
\acro{RTBP}{Restricted Three Body Problem}
\acro{SA}{Simulated Annealing}
\acro{SLS}{Space Launch System}
\acro{SNOPT}{Sparse Nonlinear OPTimizer}
\acro{SOI}{sphere of influence}
\acro{SPE}{solar particle events}
\acro{SQP}{sequential quadratic programming}
\acro{SRAG}{Space Radiation Analysis Group}
\acro{TEI}{Trans-Earth Injection}
\acro{TOF}{time of flight}
\acro{TPBVP}{Two Point Boundary Value Problem}
\acro{TMI}{Trans-Mars Injection}
\acro{VARITOP}{Variational calculus Trajectory Optimization Program}
\acro{VILM}{v-infinity leveraging maneuver}
\acro{MOI}{Mar Orbit Injection}
\acro{PCM}{Pressurized Cargo Module}
\acro{STS}{Space Transportation System}
\acro{EDS}{Earth Departure Stage}
\acro{NEO}{near-Earth asteroid}
\acro{IDC}{Integrated Design Center}
\acro{SEP}{solar-electric propulsion}
\acro{SRP}{solar radiation pressure}
\acro{NEP}{nuclear-electric propulsion}
\acro{REP}{radioisotope-electric propulsion}
\acro{DRM}{Design Reference Missions}

\acro{EDL}{entry, descent, and landing}
\acro{ASCII}{American Standard Code for Information Interchange}
\acro{AU}{Astronomical Unit}
\acro{BWG}{Beam Waveguides}
\acro{CCB}{Configuration Control Board}
\acro{CMO}{Configuration Management Office}
\acro{CODATA}{Committee on Data for Science and Technology}
\acro{DEEVE}{Dynamically Equivalent Equal Volume Ellipsoid}
\acro{DRA}{Design Reference Asteroid}
\acro{EME2000}{Earth Centered, Earth Mean Equator and Equinox of J2000 (Coordinate Frame)}
\acro{EOP}{Earth Orientation Parameters}
\acro{ET}{Ephemeris Time}
\acro{FDS}{Flight Dynamics System}
\acro{FTP}{File Transfer Protocol}
\acro{GSFC}{Goddard Space Flight Center}
\acro{PI}{Principal Investigator}
\acro{HEF}{High Efficiency}
\acro{IAG}{International Association of Geodesy}
\acro{IAU}{International Astronomical Union}
\acro{IERS}{International Earth Rotation and Reference Systems Service}
\acro{ICRF}{International Celestial Reference Frame}
\acro{ITRF}{International Terrestrial Reference System}
\acro{IOM}{Interoffice Memorandum}
\acro{JD}{Julian Date}
\acro{JPL}{Jet Propulsion Laboratory}
\acro{LM}{Lockheed Martin}
%\acro{LP150Q}{}
%\acros{LP100K}{}
\acro{MAVEN}{Mars Atmosphere and Volatile EvolutioN}
\acro{MJD}{Modified Julian Date}
\acro{MOID}{Minimum Orbit Intersection Distance}
\acro{MPC}{Minor Planet Center}
\acro{NASA}{National Aeronautics and Space Administration}
\acro{NDOSL}{\ac{NASA} Directory of Station Locations}
\acro{NEA}{near-Earth asteroid}
\acro{NEO}{near-Earth object}
\acro{NIO}{Nav IO}
\acro{OSIRIS-REx}{Origins Spectral Interpretation Resource Identification Security-Regolith Explorer}
\acro{PHA}{Potentially Hazardous Asteroid}
\acro{PHO}{Potentially Hazardous Object}
\acro{SBDB}{Small-Body Database}
\acro{SI}{International System of Units}
\acro{SPICE}{Spacecraft Planet Instrument Camera-matrix Events}
\acro{SPK}{SPICE Kernel}
\acro{SRC}{Sample Return Capsule}
\acro{SSD}{Solar System Dynamics}
\acro{STK}{Systems Tool Kit}
\acro{TAI}{International Atomic Time}
\acro{TBD}{To Be Determined}
\acro{TBR}{To Be Reviewed}
\acro{TCB}{Barycentric Coordinate Time}
\acro{TDB}{Temps Dynamiques Barycentrique, Barycentric Dynamical Time}
\acro{TDT}{Terrestrial Dynamical Time}
\acro{TT}{Terrestrial Time}
\acro{URL}{Uniform Resource Locator}
\acro{UT}{Universal Time}
\acro{UT1}{Universal Time Corrected for Polar Motion}
\acro{UTC}{Coordinated Universal Time}
\acro{USNO}{U. S. Naval Observatory}
\acro{YORP}{Yarkovsky-O'Keefe-Radzievskii-Paddack}

\acro{NLP}{nonlinear program}
\acro{MBH}{monotonic basin hopping}
\acro{MBH-C}{monotonic basin hopping with Cauchy hops}
\acro{FBS}{forward-backward shooting}
\acro{MGALT}{Multiple Gravity Assist with Low-Thrust}
\acro{MGALTS}{Multiple Gravity Assist with Low-Thrust using the Sundman transformation}
\acro{MGA-1DSM}{Multiple Gravity Assist with One Deep Space Maneuver}
\acro{MGAnDSMs}{Multiple Gravity Assist with \textit{n} Deep-Space Maneuvers using Shooting}
\acro{PSFB}{Parallel Shooting with Finite-Burn}
\acro{PSBI}{Parallel Shooting with Bounded Impulses}
\acro{FBLT}{Finite-Burn Low-Thrust}
\acro{FBLTS}{Finite-Burn Low-Thrust using the Sundman transformation}
\acro{ESA}{European Space Agency}
\acro{ACT}{Advanced Concepts Team}
\acro{IRAD}{independent research and development}
\acro{Isp}[$\text{I}_{sp}$]{specific impulse}
\acro{C3}[$C_3$]{hyperbolic excess energy}
\acro{GA}{genetic algorithm}
\acro{GALLOP}{ Gravity Assisted Low-thrust Local Optimization Program}
\acro{MALTO}{Mission Analysis Low-Thrust Optimization}
\acro{PaGMO}{Parallel Global Multiobjective Optimizer}
\acro{FRA}{feasible region analysis}
\acro{CP}{conditional penalty}
\acro{HOC}{hybrid optimal control}
\acro{HOCP}{hybrid optimal control problem}
\acro{PSO}{particle swarm optimization}
\acro{SEPTOP}{Solar Electric Propulsion Trajectory Optimization Program}
\acro{STOUR}{Satellite Tour Design Program}
\acro{STOUR-LTGA}{Satellite Tour Design Program - Low Thrust, Gravity Assist}
\acro{PaGMO}{Parallel Global Multiobjective Optimizer}
\acro{SDC}{static/dynamic control}
\acro{DDP}{Differential Dynamic Programming}
\acro{HDDP}{Hybrid Differential Dynamic Programming}
\acro{ACT}{Advanced Concepts Team}
\acro{GMAT}{General Mission Analysis Toolkit}
\acro{BOL}{beginning of life}
\acro{EOL}{end of life}
\acro{KSC}{Kennedy Space Center}
\acro{VSI}{variable \ac{Isp}}
\acro{RTG}{radioisotope thermal generator}
\acro{ASRG}{advanced Stirling radiosotope generator}
\acro{ARRM}{Asteroid Robotic Redirect Mission}
\acro{AATS}{Alternative Architecture Trade Study}
\acro{PPU}{power processing unit}
\acro{STM}{state transition matrix}
\acro{MTM}{maneuver transition matrix}
\acro{HPTM}{half-phase transition matrix}
\acro{BCI}{body-centered inertial}
\acro{BCF}{body-centered fixed}
\acro{UTTR}{Utah Test and Training Range}
\acro{EPV}{equatorial projection of $\mathbf{v}_\infty$}
\acro{KBO}{Kuiper belt object}
\acro{DSM}{deep-space maneuver}
\acro{BPT}{body-probe-thrust}
\acro{4PL}{four parameter logistic}
\acro{BCF}{body-centered fixed}
\acro{COE}{classical orbit elements}
\acro{GSL}{Gnu Scientific Library}
\acro{NEXT}{NASA's Evolutionary Xenon Thruster}

\acro{SMA}{semi-major axis}
\acro{ECC}{eccentricity}

\acro{GSAD}{Ghosh Sparse Algorithmic Differentiation}

\end{acronym}

% --------------------------------------------------------------------------------------------------------------------------
% --------------------------------------------------------------------------------------------------------------------------


\begin{abstract}
This document describes the method used by EMTG to convert a zero-sphere-of-influence flyby to a non-zero-sphere-of-influence flyby. EMTG implements this capability in the PyEMTG/HighFidelity Python scripts.

\end{abstract}

\tableofcontents

\printnomenclature

%%%%%%%%%%%%%%%%%%%%%%%%%%%%%%%%%%%%%%%%%%%%%%%%%%%%%%%%%%%%%%%%%%%%%%%%%%%%%%%
\section{ZSOI Flyby to Periapse Position and Velocity Vectors}
\label{sec:zsoi_to_rv}
%%%%%%%%%%%%%%%%%%%%%%%%%%%%%%%%%%%%%%%%%%%%%%%%%%%%%%%%%%%%%%%%%%%%%%%%%%%%%%%

The principle is to take the known quantities from the feasible \ac{ZSOI} flyby to produce initial guesses for quantities used to parameterize the non-\ac{ZSOI} flyby. The known quantities are:

\begin{itemize}
	\item $\vb{v}_{\infty}^{-}$
	\item $\vb{v}_{\infty}^{+}$
	\item $r_p$
	\item $r_{SOI}$ (set by the user for the flyby body in the \ac{EMTG} universe file)
\end{itemize}

\nomenclature{$\vb{v}_{\infty}^{-}$}{Incoming hyperbolic excess velocity vector}
\nomenclature{$\vb{v}_{\infty}^{+}$}{Outgoing hyperbolic excess velocity vector}
\nomenclature{$r_p$}{Periapse distance}
\nomenclature{$r_{SOI}$}{Sphere of influence radius}

The above data may be converted into a position and velocity at periapse of the flyby in the following manner. A unit vector pointing in the direction of periapse is obtained by

\begin{align}
	\vbh{r}_p &= \frac{\vb{v}_{\infty}^{-} - \vb{v}_{\infty}^{+}}{|| \vb{v}_{\infty}^{-} - \vb{v}_{\infty}^{+} ||}
\end{align}

\nomenclature{$\vb{r}_p$}{Position vector at periapse}

\noindent The angular momentum direction is

\begin{align}
\vbh{h} &= \frac{\vb{v}_{\infty}^{-} \times \vb{v}_{\infty}^{+}}{|| \vb{v}_{\infty}^{-} \times \vb{v}_{\infty}^{+} ||}
\end{align}

\nomenclature{$\vb{h}$}{Angular velocity vector}

\noindent The velocity direction at periapse is

\begin{align}
	\vbh{v}_p &= \frac{\vbh{h} \times \vbh{r}_p}{|| \vbh{h} \times \vbh{r}_p ||}
\end{align}

\nomenclature{$\vb{v}_p$}{Velocity vector at periapse}

\noindent From energy considerations, the velocity magnitude at periapse is

\begin{align}
	v_p &= \left( \frac{2 \mu}{r_p} + \vb{v}_{\infty}^{-T} \vb{v}_{\infty}^{+} \right)^{1/2}
\end{align}

\nomenclature{$\mu$}{Gravitational parameter of central body}
\nomenclature{$v_p$}{Magnitude of velocity at periapse}

\noindent With the direction and magnitude of position and velocity known, we finally have

\begin{align}
	\vb{r}_p &= r_p \vbh{r}_p \\
	\vb{v}_p &= v_p \vbh{v}_p
\end{align}

\noindent The preceding procedure is implemented in \ac{EMTG} in EphemerisPeggedFlybyOut::calculate\_flyby\_periapse\_state(). The flyby state is output for an ephemeris-pegged flyby in the section of the .emtg output file for the phase following the flyby.

%%%%%%%%%%%%%%%%%%%%%%%%%%%%%%%%%%%%%%%%%%%%%%%%%%%%%%%%%%%%%%%%%%%%%%%%%%%%%%%
\section{Periapse Position and Velocity Vectors to Decision Variable Guesses}
\label{sec:periapse_rv_to_dec_vars}
%%%%%%%%%%%%%%%%%%%%%%%%%%%%%%%%%%%%%%%%%%%%%%%%%%%%%%%%%%%%%%%%%%%%%%%%%%%%%%%

The Python converter scripts have the task of taking data obtained from the .emtg and .emtgopt files and converting them to guesses for the state at the \ac{SOI} boundaries, the state at periapse, and the \acp{TOF} between the \ac{SOI} boundaries and periapse (incoming and outgoing).

The periapse position and velocity vectors, which are reported in the .emtg file, are converted to Keplerian orbital elements. The true anomalies at the incoming and outgoing \ac{SOI} boundaries, as well as the times of flight between the \ac{SOI} boundaries and periapse, are calculated assuming two-body motion.

\begin{align}
	H_{SOI} &= \mathrm{acosh} \left[ \frac{1}{e} \left( \frac{r_{SOI}}{-a} + 1 \right) \right] \\
	N &= e \mathrm{sinh} H_{SOI} - H_{SOI} \\
	\label{eq:tsoi}
	\Delta t_{SOI} &= \frac{N}{ \left[ \frac{\mu}{-\left( a^3 \right)} \right]^{1/2}} \\
	\nu_{SOI} &= 2 \mathrm{atan} \left[ \sqrt{\frac{e + 1}{e - 1}} \mathrm{tanh} \left( \frac{H_{SOI}}{2} \right) \right]
\end{align}

\nomenclature{$a$}{Semimajor axis}
\nomenclature{$e$}{Eccentricity}
\nomenclature{$H$}{Hyperbolic anomaly}
\nomenclature{$t$}{Time}
\nomenclature{$\nu$}{True anomaly}
\nomenclature{$\Delta t_{SOI}$}{Time of flight from sphere of influence to periapse}

\subsection{Incoming SOI Boundary}

$\vb{v}_{\infty}^{-}$ for the \ac{ZSOI} flyby is obtained from the .emtgopt decision variables. The angular direction of $\vb{v}_{\infty}^{-}$ is

\begin{align}
v_{RA} &= \mathrm{atan2} \left( \vb{v}_{\infty, y}^{-}, \vb{v}_{\infty, x}^{-} \right) \\
v_{DEC} &= \mathrm{asin} \left( \frac{\vb{v}_{\infty, z}^{-}}{v_{\infty}^{-}} \right)
\end{align}

\noindent The Cartesian position vector at the incoming \ac{SOI} boundary (calculated from the Keplerian elements, based on the periapse state) is used to calculate the spherical coordinates at the incoming \ac{SOI} boundary:

\begin{align}
r_{RA} &= \mathrm{atan2} \left( \vb{r}_{SOI, y}, \vb{r}_{SOI, x} \right) \\
r_{DEC} &= \mathrm{asin} \left( \frac{\vb{r}_{SOI, z}}{r_{SOI}} \right)
\end{align}

The \ac{TOF} for the journey from the \ac{SOI} to periapse is taken from Eq.~\eqref{eq:tsoi}. The same amount of time is \emph{subtracted} from the \ac{TOF} of the journey that ends at the incoming \ac{SOI} boundary in order to keep the overall mission \ac{TOF} consistent.

These actions are performed in HighFidelityJourney.CreateInitialGuess().

\subsection{Periapse}

The initial guess for the spacecraft state at periapse is taken directly from the periapse state written in the .emtg file. This action is performed in HighFidelityFlybyIn.CreateInitialGuess().

\subsection{Outgoing SOI Boundary}

$\vb{v}_{\infty}^{+}$ for the \ac{ZSOI} flyby is obtained from the .emtgopt decision variables. The angular direction of $\vb{v}_{\infty}^{+}$ is

\begin{align}
v_{RA} &= \mathrm{atan2} \left( \vb{v}_{\infty, y}^{+}, \vb{v}_{\infty, x}^{+} \right) \\
v_{DEC} &= \mathrm{asin} \left( \frac{\vb{v}_{\infty, z}^{+}}{v_{\infty}^{+}} \right)
\end{align}

\noindent The Cartesian position vector at the outgoing \ac{SOI} boundary (calculated from the Keplerian elements, based on the periapse state) is used to calculate the spherical coordinates at the outgoing \ac{SOI} boundary:

\begin{align}
r_{RA} &= \mathrm{atan2} \left( \vb{r}_{SOI, y}, \vb{r}_{SOI, x} \right) \\
r_{DEC} &= \mathrm{asin} \left( \frac{\vb{r}_{SOI, z}}{r_{SOI}} \right)
\end{align}

The \ac{TOF} for the journey from periapse to the \ac{SOI} is taken from Eq.~\eqref{eq:tsoi}. The same amount of time is \emph{subtracted} from the \ac{TOF} of the journey that starts at the outgoing \ac{SOI} boundary in order to keep the overall mission \ac{TOF} consistent.

These actions are performed in HighFidelityFlybyOut.CreateInitialGuess().

\nomenclature{$v_{RA}$}{Right ascension of velocity vector}
\nomenclature{$v_{DEC}$}{Declination of velocity vector}
\nomenclature{$r_{RA}$}{Right ascension of position vector}
\nomenclature{$r_{DEC}$}{Declination of position vector}






%\bibliography{}
%\bibliographystyle{plain}

\end{document}