\documentclass[11pt]{article}

%% PACKAGES
\usepackage{graphicx}
\usepackage{verbatim}
\usepackage{url}
\usepackage{xurl}
\usepackage[printonlyused]{acronym}
\usepackage[ruled]{algorithm}
\usepackage{amsmath,amssymb,amsfonts,amsthm}
\usepackage{overpic}
\usepackage{calc}
\usepackage{color}
%\usepackage{times}
%\usepackage{ragged2e}
\usepackage[margin=1.0in]{geometry}
\usepackage[colorlinks=false]{hyperref}
\usepackage{textcomp}
\usepackage{cite}
\usepackage{mdwlist}
\usepackage{subfiles}
\usepackage{enumitem}
\usepackage{calc}
\usepackage{array}
\usepackage{units}
\usepackage{arydshln,leftidx,mathtools}
\usepackage[caption=false,font=footnotesize]{subfig}
\usepackage{relsize}
\usepackage{float}
\usepackage{makecell}

\usepackage{algorithm}
\usepackage[noend]{algpseudocode}

\usepackage{tabularx}

\makeatletter
\let\@tmp\@xfloat
\usepackage{fixltx2e}
\let\@xfloat\@tmp
\makeatother

\usepackage[subfigure]{tocloft}
\usepackage[singlespacing]{setspace}
%\usepackage[nodisplayskipstretch]{setspace}
%\setstretch{1.0}

%\renewcommand\cftsecafterpnum{\vskip\baselineskip}
%\renewcommand\cftsubsecafterpnum{\vskip\baselineskip}
%\renewcommand\cftsubsubsecafterpnum{\vskip\baselineskip}

%\usepackage{mathtools}
%\usepackage[framed]{mcode}

\usepackage{pgfplots}

\usepackage{cancel}

\usepackage{tikz}
\usetikzlibrary{calc,patterns,decorations.pathmorphing,decorations.markings,fit,backgrounds}

\usepackage[strict]{changepage} %use to manually place figs/tables to get them within the margins

\makeatletter
\g@addto@macro\normalsize{%
	\setlength\abovedisplayskip{0.25pt}
	\setlength\belowdisplayskip{0.25pt}
	\setlength\abovedisplayshortskip{0.25pt}
	\setlength\belowdisplayshortskip{0.25pt}
}
\makeatother



\setlength{\parskip}{\baselineskip}

%% GRAPHICS PATH
\graphicspath{{./pictures/pdf/}{./pictures/ps/}{./pictures/png/}}

%% TODO
\newcommand{\todo}[1]{\vspace{5 mm}\par \noindent \framebox{\begin{minipage}[c]{0.98 \columnwidth} \ttfamily\flushleft \textcolor{red}{#1}\end{minipage}}\vspace{5 mm}\par}

%% MACROS
\providecommand{\abs}[1]{\lvert#1\rvert}
\providecommand{\norm}[1]{\lVert#1\rVert}
\providecommand{\dualnorm}[1]{\norm{#1}_\ast}
\providecommand{\set}[1]{\lbrace\,#1\,\rbrace}
\providecommand{\cset}[2]{\lbrace\,{#1}\nobreak\mid\nobreak{#2}\,\rbrace}
\providecommand{\onevect}{\mathbf{1}}
\providecommand{\zerovect}{\mathbf{0}}
\providecommand{\field}[1]{\mathbb{#1}}
\providecommand{\C}{\field{C}}
\providecommand{\R}{\field{R}}
\providecommand{\polar}{\triangle}
\providecommand{\Cspace}{\mathcal{Q}}
\providecommand{\Fspace}{\mathcal{F}}
\providecommand{\free}{\text{\{}\mathsf{free}\text{\}}}
\providecommand{\iff}{\Leftrightarrow}
\providecommand{\qstart}{q_\text{initial}}
\providecommand{\qgoal}{q_\text{final}}
\providecommand{\contact}[1]{\Cspace_{#1}}
\providecommand{\feasible}[1]{\Fspace_{#1}}
\providecommand{\prob}[2]{p(#1|#2)}
\providecommand{\prior}[1]{p(#1)}
\providecommand{\Prob}[2]{P(#1|#2)}
\providecommand{\Prior}[1]{P(#1)}
\providecommand{\parenth}[1] {\left(#1\right)}
\providecommand{\braces}[1] {\left\{#1\right\}}
\providecommand{\micron}{\hbox{\textmu m}}

%% MATH FUNCTION NAMES
\DeclareMathOperator{\conv}{conv}
\DeclareMathOperator{\cone}{cone}
\DeclareMathOperator{\homog}{homog}
\DeclareMathOperator{\domain}{dom}
\DeclareMathOperator{\range}{range}
\DeclareMathOperator{\argmax}{arg\,max}
\DeclareMathOperator{\argmin}{arg\,min}
\DeclareMathOperator{\area}{area}
\DeclareMathOperator{\sign}{sign}
\DeclareMathOperator{\mathspan}{span}
\DeclareMathOperator{\sn}{sn}
\DeclareMathOperator{\cn}{cn}
\DeclareMathOperator{\dn}{dn}
\DeclareMathOperator*{\minimize}{minimize}

\DeclareMathOperator{\atan2}{atan2}

\newtheorem{theorem}{Theorem}
\newtheorem{lemma}[theorem]{Lemma}

%\setlength{\RaggedRightParindent}{2em}
%\setlength{\RaggedRightRightskip}{0pt plus 3em}
%\pagestyle{empty}

\newcommand{\acposs}[1]{%
	\expandafter\ifx\csname AC@#1\endcsname\AC@used
	\acs{#1}'s%
	\else
	\aclu{#1}'s (\acs{#1}'s)%
	\fi
}


\title{{\Huge PyEMTG Documentation Readme}}
\vspace{0.5cm}
\author
{
	Noble Hatten\thanks{Aerospace Engineer, NASA Goddard Space Flight Center, Code 595}
	Edwin Dove\thanks{Aerospace Engineer, NASA Goddard Space Flight Center, Code 595}
}
\vspace{0.5cm}

\date{}

\begin{document}
	
\begin{titlepage}
	\maketitle
	%\thispagestyle{empty}
	\begin{table}[H]
		\centering
		\begin{tabularx}{\textwidth}{|l|l|X|}
			\hline
			\textbf{Revision Date} & \textbf{Author} & \textbf{Description of Change} \\ \hline
			\date{July 22, 2022} & Noble Hatten & Initial creation and migration of markdown content into this document.\\ \hline
			\date{October 25, 2023} & Edwin Dove & Updated instructions to point to the new documentation location. \\ 
			\hline
		\end{tabularx}
	\end{table}
\end{titlepage}

\newpage
\tableofcontents
\thispagestyle{empty}
\newpage

\clearpage
\setcounter{page}{1}




\section*{List of Acronyms}
\begin{acronym}
%To define the acronym and include it in the list of acronyms: \acro{acronym}{definition}
%To define the acronym and exclude it from the list of acronyms:  \acro{acronym}{definition}
%
%\ac{acronym} Expand and identify the acronym the first time; use only the acronym thereafter
%\acf{acronym} Use the full name of the acronym.
%\{acronym} Use the acronym, even before the first corresponding \ac command
%\acl{acronym}  Expand the acronym without using the acronym itself.
%
%

\acro{ACS}{attitude control system}
\acro{ACO}{Ant Colony Optimization}
\acro{AD}{Automatic Differentiation}
\acro{ADL}{Architecture Design Laboratory}
\acro{AES}{Advanced Exploration Systems}
\acro{AGA}{aerogravity assist}
\acro{ALARA}{As Low As Reasonably Achievable}
\acro{API}{application programming interface}
\acro{BB}{branch and bound}
\acro{BVP}{Boundary Value Problem}
\acro{CATO}{Computer Algorithm for Trajectory Optimization}
\acro{CL}{confidence level}
\acro{CONOPS}{concept of operations}
\acro{COV}{Calculus of Variations}
\acro{D/AV}{Descent/Ascent Vehicle}
\acro{DE}{Differential Evolution}
\acro{DLA}{Declination of Launch Asymptote}
\acro{RLA}{Right Ascension of Launch Asymptote}
\acro{RA}{right ascension}
\acro{DEC}{declination}
\acro{DPTRAJ/ODP}{Double Precision Trajectory and Orbit Determination Program}
\acro{DSH}{Deep Space Habitat}
\acro{DSN}{Deep Space Network}
\acro{DSMPGA}{Dynamic-Size Multiple Population Genetic Algorithm}
\acro{EB}{Evolutionary Branching}
\acro{ECLSS}{environmental control and life support system}
\acro{ELV}{expendable launch vehicle}
\acro{EMME}{Earth to Mars, Mars to Earth}
\acro{EMMVE}{Earth to Mars, Mars to Venus to Earth}
\acro{EMTG}{Evolutionary Mission Trajectory Generator}
\acro{EVMME}{Earth to Venus to Mars, Mars to Earth}
\acro{EVMMVE}{Earth to Venus to Mars, Mars to Venus to Earth}
\acro{ERRV}{Earth Return Re-entry Vehicle}
\acro{FISO}{Future In-Space Operations}
\acro{FMT}{Fast Mars Transfer}
\acro{GASP}{Gravity Assist Space Pruning}
\acro{GCR}{galactic cosmic radiation}
\acro{GRASP}{Greedy Randomized Adaptive Search Procedure}
\acro{GSFC}{Goddard Space Flight Center}
\acro{GTOC}{Global Trajectory Optimization Competition}
\acro{GTOP}{Global Trajectory Optimization Problem}
\acro{HAT}{Human Architecture Team}
\acro{HGGA}{Hidden Genes Genetic Algorithm}
\acro{IMLEO}{Initial Mass in \acl{LEO}}
\acro{IPOPT}{Interior Point OPTimizer}
\acro{ISS}{International Space Station}
\acro{JHUAPL}{Johns Hopkins University Applied Physics Laboratory}
\acro{JSC}{Johnson Space Center}
\acro{KKT}{Karush-Kuhn-Tucker}
\acro{LEO}{Low Earth Orbit}
\acro{LRTS}{lazy race tree search}
\acro{MONTE}{Mission analysis, Operations, and Navigation Toolkit Environment}
\acro{MCTS}{Monte Carlo tree search}
\acro{MGA}{Multiple Gravity Assist}
\acro{MIRAGE}{Multiple Interferometric Ranging Analysis using GPS Ensemble}
\acro{MOGA}{Multi-Objective Genetic Algorithm}
\acro{MOSES}{Multiple Orbit Satellite Encounter Software}
\acro{MPI}{message passing interface}
\acro{MPLM}{Multi-Purpose Logistics Module}
\acro{MSFC}{Marshall Space Flight Center}
\acro{NELLS}{NASA Exhaustive Lambert Lattice Search}
\acro{NSGA}{Non-Dominated Sorting Genetic Algorithm}
\acro{NSGA-II}{Non-Dominated Sorting Genetic Algorithm II}
\acro{NHATS}{Near-Earth Object Human Space Flight Accessible Targets Study}
\acro{NTP}{Nuclear Thermal Propulsion}
\acro{OD}{orbit determination}
\acro{OOS}{On-Orbit Staging}
\acro{PCC}{Pork Chop Contour}
\acro{PEL}{permissible exposure limits}
\acro{PLATO}{PLAnetary Trajectory Optimization}
\acro{REID}{risk of exposure-induced death}
\acro{RTBP}{Restricted Three Body Problem}
\acro{SA}{Simulated Annealing}
\acro{SLS}{Space Launch System}
\acro{SNOPT}{Sparse Nonlinear OPTimizer}
\acro{SOI}{sphere of influence}
\acro{SPE}{solar particle events}
\acro{SQP}{sequential quadratic programming}
\acro{SRAG}{Space Radiation Analysis Group}
\acro{TEI}{Trans-Earth Injection}
\acro{TOF}{time of flight}
\acro{TPBVP}{Two Point Boundary Value Problem}
\acro{TMI}{Trans-Mars Injection}
\acro{VARITOP}{Variational calculus Trajectory Optimization Program}
\acro{VILM}{v-infinity leveraging maneuver}
\acro{MOI}{Mar Orbit Injection}
\acro{PCM}{Pressurized Cargo Module}
\acro{STS}{Space Transportation System}
\acro{EDS}{Earth Departure Stage}
\acro{NEO}{near-Earth asteroid}
\acro{IDC}{Integrated Design Center}
\acro{SEP}{solar-electric propulsion}
\acro{SRP}{solar radiation pressure}
\acro{NEP}{nuclear-electric propulsion}
\acro{REP}{radioisotope-electric propulsion}
\acro{DRM}{Design Reference Missions}

\acro{EDL}{entry, descent, and landing}
\acro{ASCII}{American Standard Code for Information Interchange}
\acro{AU}{Astronomical Unit}
\acro{BWG}{Beam Waveguides}
\acro{CCB}{Configuration Control Board}
\acro{CMO}{Configuration Management Office}
\acro{CODATA}{Committee on Data for Science and Technology}
\acro{DEEVE}{Dynamically Equivalent Equal Volume Ellipsoid}
\acro{DRA}{Design Reference Asteroid}
\acro{EME2000}{Earth Centered, Earth Mean Equator and Equinox of J2000 (Coordinate Frame)}
\acro{EOP}{Earth Orientation Parameters}
\acro{ET}{Ephemeris Time}
\acro{FDS}{Flight Dynamics System}
\acro{FTP}{File Transfer Protocol}
\acro{GSFC}{Goddard Space Flight Center}
\acro{PI}{Principal Investigator}
\acro{HEF}{High Efficiency}
\acro{IAG}{International Association of Geodesy}
\acro{IAU}{International Astronomical Union}
\acro{IERS}{International Earth Rotation and Reference Systems Service}
\acro{ICRF}{International Celestial Reference Frame}
\acro{ITRF}{International Terrestrial Reference System}
\acro{IOM}{Interoffice Memorandum}
\acro{JD}{Julian Date}
\acro{JPL}{Jet Propulsion Laboratory}
\acro{LM}{Lockheed Martin}
%\acro{LP150Q}{}
%\acros{LP100K}{}
\acro{MAVEN}{Mars Atmosphere and Volatile EvolutioN}
\acro{MJD}{Modified Julian Date}
\acro{MOID}{Minimum Orbit Intersection Distance}
\acro{MPC}{Minor Planet Center}
\acro{NASA}{National Aeronautics and Space Administration}
\acro{NDOSL}{\ac{NASA} Directory of Station Locations}
\acro{NEA}{near-Earth asteroid}
\acro{NEO}{near-Earth object}
\acro{NIO}{Nav IO}
\acro{OSIRIS-REx}{Origins Spectral Interpretation Resource Identification Security-Regolith Explorer}
\acro{PHA}{Potentially Hazardous Asteroid}
\acro{PHO}{Potentially Hazardous Object}
\acro{SBDB}{Small-Body Database}
\acro{SI}{International System of Units}
\acro{SPICE}{Spacecraft Planet Instrument Camera-matrix Events}
\acro{SPK}{SPICE Kernel}
\acro{SRC}{Sample Return Capsule}
\acro{SSD}{Solar System Dynamics}
\acro{STK}{Systems Tool Kit}
\acro{TAI}{International Atomic Time}
\acro{TBD}{To Be Determined}
\acro{TBR}{To Be Reviewed}
\acro{TCB}{Barycentric Coordinate Time}
\acro{TDB}{Temps Dynamiques Barycentrique, Barycentric Dynamical Time}
\acro{TDT}{Terrestrial Dynamical Time}
\acro{TT}{Terrestrial Time}
\acro{URL}{Uniform Resource Locator}
\acro{UT}{Universal Time}
\acro{UT1}{Universal Time Corrected for Polar Motion}
\acro{UTC}{Coordinated Universal Time}
\acro{USNO}{U. S. Naval Observatory}
\acro{YORP}{Yarkovsky-O'Keefe-Radzievskii-Paddack}

\acro{NLP}{nonlinear program}
\acro{MBH}{monotonic basin hopping}
\acro{MBH-C}{monotonic basin hopping with Cauchy hops}
\acro{FBS}{forward-backward shooting}
\acro{MGALT}{Multiple Gravity Assist with Low-Thrust}
\acro{MGALTS}{Multiple Gravity Assist with Low-Thrust using the Sundman transformation}
\acro{MGA-1DSM}{Multiple Gravity Assist with One Deep Space Maneuver}
\acro{MGAnDSMs}{Multiple Gravity Assist with \textit{n} Deep-Space Maneuvers using Shooting}
\acro{PSFB}{Parallel Shooting with Finite-Burn}
\acro{PSBI}{Parallel Shooting with Bounded Impulses}
\acro{FBLT}{Finite-Burn Low-Thrust}
\acro{FBLTS}{Finite-Burn Low-Thrust using the Sundman transformation}
\acro{ESA}{European Space Agency}
\acro{ACT}{Advanced Concepts Team}
\acro{IRAD}{independent research and development}
\acro{Isp}[$\text{I}_{sp}$]{specific impulse}
\acro{C3}[$C_3$]{hyperbolic excess energy}
\acro{GA}{genetic algorithm}
\acro{GALLOP}{ Gravity Assisted Low-thrust Local Optimization Program}
\acro{MALTO}{Mission Analysis Low-Thrust Optimization}
\acro{PaGMO}{Parallel Global Multiobjective Optimizer}
\acro{FRA}{feasible region analysis}
\acro{CP}{conditional penalty}
\acro{HOC}{hybrid optimal control}
\acro{HOCP}{hybrid optimal control problem}
\acro{PSO}{particle swarm optimization}
\acro{SEPTOP}{Solar Electric Propulsion Trajectory Optimization Program}
\acro{STOUR}{Satellite Tour Design Program}
\acro{STOUR-LTGA}{Satellite Tour Design Program - Low Thrust, Gravity Assist}
\acro{PaGMO}{Parallel Global Multiobjective Optimizer}
\acro{SDC}{static/dynamic control}
\acro{DDP}{Differential Dynamic Programming}
\acro{HDDP}{Hybrid Differential Dynamic Programming}
\acro{ACT}{Advanced Concepts Team}
\acro{GMAT}{General Mission Analysis Toolkit}
\acro{BOL}{beginning of life}
\acro{EOL}{end of life}
\acro{KSC}{Kennedy Space Center}
\acro{VSI}{variable \ac{Isp}}
\acro{RTG}{radioisotope thermal generator}
\acro{ASRG}{advanced Stirling radiosotope generator}
\acro{ARRM}{Asteroid Robotic Redirect Mission}
\acro{AATS}{Alternative Architecture Trade Study}
\acro{PPU}{power processing unit}
\acro{STM}{state transition matrix}
\acro{MTM}{maneuver transition matrix}
\acro{HPTM}{half-phase transition matrix}
\acro{BCI}{body-centered inertial}
\acro{BCF}{body-centered fixed}
\acro{UTTR}{Utah Test and Training Range}
\acro{EPV}{equatorial projection of $\mathbf{v}_\infty$}
\acro{KBO}{Kuiper belt object}
\acro{DSM}{deep-space maneuver}
\acro{BPT}{body-probe-thrust}
\acro{4PL}{four parameter logistic}
\acro{BCF}{body-centered fixed}
\acro{COE}{classical orbit elements}
\acro{GSL}{Gnu Scientific Library}
\acro{NEXT}{NASA's Evolutionary Xenon Thruster}

\acro{SMA}{semi-major axis}
\acro{ECC}{eccentricity}

\acro{GSAD}{Ghosh Sparse Algorithmic Differentiation}

\end{acronym}

% --------------------------------------------------------------------------------------------------------------------------
% --------------------------------------------------------------------------------------------------------------------------


%%%%%%%%%%%%%%%%%%%%%%%%
\section{Introduction}
\label{sec:introduction}
%%%%%%%%%%%%%%%%%%%%%%%%

The purpose of this document is to describe:

\begin{itemize}
	\item The available documentation for PyEMTG (Section~\ref{sec:pyemtg_documentation}).
	\item How to add new documentation to PyEMTG (Section~\ref{sec:creating_new_pyemtg_rst_documentation}).
\end{itemize}

%%%%%%%%%%%%%%%%%%%%%%%%
\section{PyEMTG Documentation}
\label{sec:pyemtg_documentation}
%%%%%%%%%%%%%%%%%%%%%%%%

PyEMTG documentation is located in the \ac{EMTG} git repo. 
The following notation is used in this documentation to represent folder locations: 
\begin{itemize}
	\item \textbf{\textless EMTG\_root\_dir\textgreater} - EMTG git repo root directory
	\item \textbf{\textless PyEMTG-autogen\_dir\textgreater} - root directory of the auto generated PyEMTG documentation and sphinx configuration files: \\ \textbf{\textless EMTG\_root\_dir\textgreater}/docs/1\_Developers/python\_modules/

\end{itemize}

\noindent The PyEMTG documentation consists of:
\begin{itemize}
	\item A \LaTeX-based user guide located in \textbf{\textless EMTG\_root\_dir\textgreater}/docs/0\_Users/PyEMTG\_User\_GuideUser/
	\item Auto-generated \ac{rst} documentation based on specially formatted comments in PyEMTG .py scripts. \ac{rst} files may be used to produce multiple formats of end-user documentation. The \ac{EMTG} repo contains html versions of this documentation that may be accessed via \textbf{\textless PyEMTG-autogen\_dir\textgreater}/\_autogendocs/html/index.html. After opening this file in a web browser, the browser may be used to navigate between different pages of documentation.
\end{itemize}

%%%%%%%%%%%%%%%%%%%%%%%%
\section{Creating New PyEMTG Restructured Text Documentation}
\label{sec:creating_new_pyemtg_rst_documentation}
%%%%%%%%%%%%%%%%%%%%%%%%

%%%%%%%%%%%%%%%%%%%%%%%%
\subsection{Prequisites}
\label{sec:prequisites_for_creating_rst_docs}
%%%%%%%%%%%%%%%%%%%%%%%%

PyEMTG documentation is generated using the Sphinx Python Documentation Generator (\url{https://www.sphinx-doc.org/en/master/index.html}). A user must have the sphinx package installed in their Python environment in order to build the documentation.

%%%%%%%%%%%%%%%%%%%%%%%%
\subsection{Updating Documentation}
\label{sec:updating_autogen_documentation}
%%%%%%%%%%%%%%%%%%%%%%%%

Documentation may be added directly to Python source files. For examples of the syntax of adding documentation, see \textbf{\textless EMTG\_root\_dir\textgreater}/PyEMTG/Journey.py and web sites like \url{https://eikonomega.medium.com/getting-started-with-sphinx-autodoc-part-1-2cebbbca5365}.

\noindent If updating auto-generated documentation for the python files in \textbf{\textless EMTG\_root\_dir\textgreater}/PyEMTG/ that already contains the same names as the rst files in the \\ \textbf{\textless PyEMTG-autogen\_dir\textgreater}/ subdirectories, then all a user needs to do for updates to appear in the html documents is to rebuild the documentation. See Section~\ref{sec:building_rst_documentation}.

\noindent If adding a new subdirectory to the \textbf{\textless EMTG\_root\_dir\textgreater}/PyEMTG/ directory, the following steps must be followed prior to rebuilding in order for documentation to be produced for Python files in the new subdirectory:

\begin{enumerate}
	\item Create a directory in \textbf{\textless PyEMTG-autogen\_dir\textgreater}/ with the name of the new directory that was created in \textbf{\textless EMTG\_root\_dir\textgreater}/PyEMTG/.
	\item Open a terminal with access to the Python environment with sphinx.
	\item In the terminal, execute \\ \texttt{sphinx-apidoc -o \textbf{\textless PyEMTG-autogen\_dir\textgreater}/<name of your new directory> \\ \textbf{\textless EMTG\_root\_dir\textgreater}/PyEMTG/<name of your new directory>/}
	\item Copy the index.rst file from a pre-existing \\ \textbf{\textless PyEMTG-autogen\_dir\textgreater}/ subdirectory to \\ \textbf{\textless PyEMTG-autogen\_dir\textgreater}/\texttt{<name of your new directory>}.
	\item Edit \textbf{\textless PyEMTG-autogen\_dir\textgreater}/\texttt{<name of your new directory>}/index.rst so the title header text above the equal (=) characters reads "\texttt{<name of your new directory>} documentation".
	\item Add \texttt{sys.path.insert(0, os.path.abspath('../../../PyEMTG/<name of your new directory>/'))} to the \textbf{\textless PyEMTG-autogen\_dir\textgreater}/docs/\textless dir\textgreater/conf.py file.
	\item Add \texttt{<name of your new directory>/index.rst} to the list of .rst files in /PyEMTG/docs/index.rst.
	\item In the terminal, navigate to \textbf{\textless PyEMTG-autogen\_dir\textgreater}/.
	\item In the terminal, execute \texttt{make html}. \\ \textit{There may be warnings but hopefully there are no errors.}
\end{enumerate}

\noindent Note: These instructions were adapted from \url{https://dev.to/dev0928/how-to-generate-professional-documentation-with-sphinx-4n78} and \url{https://stackoverflow.com/questions/13116155/sphinx-apidoc-usage-multiple-source-python-directories}.

\noindent If \textbf{\textless EMTG\_root\_dir\textgreater}/PyEMTG/\texttt{<name of preexising directory>} already exists in the \textbf{\textless PyEMTG-autogen\_dir\textgreater}/ directory and documentation is needed for a new python module, the following steps provides guidance producing the documentation:
\begin{enumerate}
	\item Open a terminal with access to the Python environment with sphinx.
	\item In the terminal, execute \\ \texttt{sphinx-apidoc -o \textbf{\textless PyEMTG-autogen\_dir\textgreater}/<name of preexising directory> \\ \textbf{\textless EMTG\_root\_dir\textgreater}/PyEMTG/<name of preexising directory>/}
	\item In the terminal, navigate to \textbf{\textless PyEMTG-autogen\_dir\textgreater}/.
	\item In the terminal, execute \texttt{make html}. \\ \textit{There may be warnings but hopefully there are no errors.}
\end{enumerate}

%%%%%%%%%%%%%%%%%%%%%%%%
\subsection{Building the Documentation}
\label{sec:building_rst_documentation}
%%%%%%%%%%%%%%%%%%%%%%%%

\begin{enumerate}
	\item Open a terminal with access to the Python environment with sphinx.
	\item Navigate to the \textbf{\textless PyEMTG-autogen\_dir\textgreater}/ directory.
	\item In the terminal, execute \texttt{make html} to produce the html-based documentation.
\end{enumerate}


\end{document}