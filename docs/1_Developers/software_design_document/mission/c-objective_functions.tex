\chapter{Objective Functions}
\label{chap:objective-functions}

\hl{Jacob}

Describe each objective function.

\section{ObjectiveFunctionBase}
\label{sec:objectivefunctionbase}

\section{ObjectiveFunctionFactory}
\label{sec:objectivefunctionfactory}

\section{MaximizeDistanceFromCentralBodyObjective}
\label{sec:maximizedistancefromcentralbodyobjective}

This objective function maximizes the distance from the central body,

\begin{equation}
	J = \sqrt{x_{cb}^2 + y_{cb}^2 + z_{cb}^2}
	\label{eq:maximizedistancefromcentralbodyobjective}
\end{equation}

The user specifies which journey the objective function applies to and the objective function is applied to the arrival event at the end of that journey.

For example, the user may wish to deflect a hazardous asteroid from hitting the Earth. The user would design a mission to the asteroid, then perform an \texttt{EphemerisPeggedMomentumTransfer} at the asteroid to simulate a kinetic impactor transferring momentum from the spacecraft to the asteroid. The remainder of the mission journeys and phases would track the motion of the post-deflection asteroid. The final journey would occur inside the Earth's sphere of influence and would end with a \texttt{PeriapseFlybyIn} event to simulate the asteroid's closest approach to the Earth. The user would then employ the \texttt{MaximizeDistanceFromCentralBodyObjective} to instruct EMTG to maximize the periapse distance.

A second example would be if the user wants to design mission to escape from the central body (\textit{i.e.} the Sun or the Earth). This mission would end with a \texttt{FreePointLTRendezvous} or \texttt{FreePointIntercept} and use this same objective function.